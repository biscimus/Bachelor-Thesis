The following section provides an overview of the related work regarding the thesis, primarily focusing on the fields of data-aware process mining, translucent event logs, and predictive process monitoring.

\section{Data-Aware Process Mining}

Conventional process mining algorithms exclusively focus on the control-flow aspect of the process, i.e., the activity attribute of the event log, thereby ignoring the data attributes the event log provides. Data-aware process mining, on the other hand, attempts to incorporate both the control-flow and data attributes of the event log. The prevalent repertoire is to discover decision points in the model and to annotate them with guard functions, which in turn are discovered with decision trees. This method of process enhancement is called \emph{decision mining} \cite{decision-mining-in-prom,decision-mining-in-business-processes,data-aware-process-mining} and is a well-established field of research within process mining. In \cite{data-aware-process-mining}, Petri nets with a data setting \emph{(DPN)} are proposed, thereby expanding the notation of Petri nets. While previous papers worked with the assumption of deterministic, mutual exclusive transition behavior in decision points, they do not take into account how certain decisions cannot be modeled in a dichotomous manner. Often, one needs a softer classification assumption, stating that data attributes affect the decision probabilistically. A further extension of the concept of data-annotated Petri nets occurs in \cite{sldpn} by integrating stochastic information into the model. In the paper, the authors introduce the concept of Stochastic Labeled Data Petri Nets \emph{SLDPNs} and propose a method to generate an SLDPN from a Petri net and an event log. Each transition will be mapped with its own weight function learned with the activation instances of individual transitions using logistic regression.

\section{Translucent Event Logs}

\textcolor{red}{Add the new paper ActivityGen as soon as it's published}

Little research has been performed on the topic of translucent event logs. Being a relatively young concept in the field of process mining, they were first hinted in 2018 by \cite{lucency-first-paper}, where a possible event log revealing the set of enabled activities is mentioned. \cite{Translucent-event-logs-first-paper} formally introduces and defines translucent event logs and relates concepts of lucency and translucency by showing that a lucent process model can be rediscovered by using a translucent event log retrieved from the model.

Methods of creating translucent event logs are first discussed in \cite{creating-translucent-event-logs}. Here, a system's screenshot is matched with the labelled activity pattern and annotated with the corresponding activities. Furthermore, a model-based approach is introduced by replaying the event log on the model and annotating enabled activities. In \cite{translucent-activity-relationships}, the authors demonstrate how using information of enabled activities can be utilized to discover a process model of equipotential quality in comparison to conventional process discovery algorithms. This is done by extending the Inductive Miner \cite{inductive-miner} using newly defined translucent activity relationships. Lastly, a precision measure between a Petri net and a translucent event log is formally introduced in \cite{translucent-precision} by comparing log-enabled activities and model-enabled activities.

\section{Predictive Process Monitoring}

\textcolor{red}{TODO: Make a clear distinction between the current section and methods used data-aware process mining, e.g., decision mining!}

\begin{itemize}
    \item Explain the term and list previous different approaches
    \item Lin et al. \cite{predictive-process-monitoring-rnn}: Next event and next data attribute prediction using RNNs
    \item Gunnarsson et al. \cite{predictive-process-monitoring-lstm}: Remaining trace and runtime prediction using LSTMs
    \item Bukhsh et al. \cite{predictive-process-monitoring-transformer}: Next activity, next event time, and remaining time prediction using transformer architecture
    \item General overview of the field is given in \cite{predictive-process-monitoring}.
\end{itemize}