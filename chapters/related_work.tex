The following section provides an overview of the related work regarding the thesis, primarily focusing on the fields of data-aware process mining, translucent event logs, and predictive process monitoring.

\section{Data-Aware Process Mining}

Conventional process mining algorithms exclusively focus on the control-flow aspect of the process, i.e., the activity attribute of the event log, thereby ignoring the data attributes the event log provides. Data-aware process mining, on the other hand, attempts to incorporate both the control-flow and data attributes of the event log. 

The prevalent repertoire is to discover decision points in the model and to annotate them with guard functions, which in turn are discovered with decision trees. This method of process enhancement is called \emph{decision mining} \cite{decision-mining-in-prom,decision-mining-in-business-processes,data-aware-process-mining} and is a well-established field of research within process mining. In \cite{data-aware-process-mining}, Petri nets with a data setting \emph{(DPN)} are proposed, thereby expanding the notation of Petri nets. While previous papers worked with the assumption of deterministic, mutual exclusive transition behavior in decision points, they do not take into account how certain decisions cannot be modeled in a dichotomous manner. Often, one needs a softer classification assumption, stating that data attributes affect the decision probabilistically.

A further extension of the concept of data-annotated Petri nets occurs in \cite{sldpn} by integrating stochastic information into the model. In the paper, the authors introduce the concept of Stochastic Labeled Data Petri Nets \emph{SLDPNs} and propose a method to generate an SLDPN from a Petri net and an event log. Each transition will be mapped with its own weight function learned with the activation instances of individual transitions using logistic regression.

\section{Translucent Event Logs}

Little research has been performed on the topic of translucent event logs. Being a relatively young concept in the field of process mining, they were first hinted in 2018 by \cite{lucency-first-paper}, where a possible event log revealing the set of enabled activities is mentioned. The concept and definition of translucent event logs are formally introduced in \cite{Translucent-event-logs-first-paper}, where it further relates concepts of lucency and translucency by showing that a lucent process model can be rediscovered by using a translucent event log retrieved from the model.

Methods of creating translucent event logs are first discussed in \cite{creating-translucent-event-logs}. Here, a system's screenshot is matched with the labelled activity pattern and annotated with the corresponding activities. Furthermore, a model-based approach is introduced by replaying the event log on the model and annotating enabled activities. In \cite{translucent-activity-relationships}, the authors demonstrate how using information of enabled activities can be utilized to discover a process model of equipotential quality in comparison to conventional process discovery algorithms. This is done by extending the Inductive Miner \cite{inductive-miner} using newly defined translucent activity relationships. A precision measure between a Petri net and a translucent event log is formally introduced in \cite{translucent-precision} by comparing log-enabled activities and model-enabled activities. Lastly, \cite{activity-gen} proposes a more enhanced method succeeding \cite{creating-translucent-event-logs} by detecting GUI elements from screenshots and matching them with the activity name database, thereby extending the user interaction log with translucent information.

\section{Predictive Process Monitoring}

The term predictive process monitoring was first proposed by \cite{predictive-monitoring-of-business-processes} and refers to the field of process mining aiming to make predictions on future outcomes still during the process execution phase, e.g., next activity, remaining time, numerical value of a data attribute, etc. Although it shares common aspects with the aforementioned field of data-aware process mining in the sense that they both focus on the data attributes. However, the core difference lies on its objective, where predictive process monitoring focuses more on the prediction of future events rather than the enhancement of the process model itself.

As a result, despite the presence of model-based approaches, we see increasing tendencies of using a model-free approach in this field of research. The usage of linear temporal logic (LTL) \cite{ltl} in \cite{predictive-monitoring-of-business-processes} is a notable example. More novel approaches include the usage of deep learning architectures. Next event and next data attribute prediction using recurrent neural networks (RNNs) \cite{rnn} are proposed in \cite{predictive-process-monitoring-rnn}. Furthermore, \cite{predictive-process-monitoring-lstm} introduces the usage of long short-term memory (LSTM) networks \cite{lstm} for remaining trace and runtime prediction. Finally, \cite{predictive-process-monitoring-transformer} attempts to make next activity, next event time, and remaining time predictions using the transformer architecture \cite{attention-is-all-you-need}. Looking at the chronological trend, It is noteworthy that the artificial intelligence models used in the related works directly correspond to the advances in the field of deep-learning. As new models are introduced, they are rapidly adapted and tested in the field of predictive process monitoring.

A general overview of the field is given in \cite{predictive-process-monitoring} and \cite{predictive-monitoring-of-business-processes}.

