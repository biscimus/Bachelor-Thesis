\begin{comment}
    In the preliminaries you present well-understood mathematical concepts that you need in your thesis.
For example, you can define the natural numbers as $\mathbb{N}{=}\{0,1,2,...\}$, and, correspondingly $\mathbb{N}^+{=}\mathbb{N}{\setminus}\{0\}$.
A preliminary notion is either a well-defined commonly understood mathematical notion, e.g., sets, multisets, graphs, sequences, Petri nets, \dots, or, it is a concept clearly defined in another paper, i.e., you just adopt the notation (or a slight variation thereof).
\emph{Any concept you use should be defined in your thesis}.
You should never write: \enquote{We use Workflow nets, a definition of these can be found here [X]}.
If you use it, explain it.

Concepts that are unique to your approach are not part of the preliminaries, i.e., they are described in the approach section itself.

Some useful tips:
\begin{itemize}
    \item When introducing a complex concept, use the following structure (always works):
    \begin{itemize}
        \item Explain the concept informally.
        \item Provide a formal definition of the concept.
        \item Provide an example, using the formal definition, of the concept.
    \end{itemize}
    In your examples, try to be \emph{as visual as you can}, often, an image says more than 5 pages of text.
    \item use commands, e.g., \texttt{$\backslash$newcommand\{$\backslash$naturals\}\{$\backslash$ensuremath\{$\backslash$mathbb\{N\}\}\}}
\end{itemize}
\end{comment}

\begin{definition}[Petri net]
\label{def:petrinet}
    Let $P, T$ be finite, disjoint sets, where $T$ is a set of \emph{places} and $T$ set of \emph{transitions}. A \emph{Petri net} is a triple $N = (P, T, F)$, where $F \subseteq (P \times T) \cup (T \times P)$ is a set of directed arcs between places and transitions.
\end{definition}

\begin{definition}[Marked Petri net]
    Let $N = (P, T, F)$ be a Petri net. A \emph{marked Petri net} is an ordered pair $(N, M)$ where $M \in \mathbb{B}(P)$ is a multiset over P.
\end{definition}

\begin{definition}[Labeled Petri net]
    
\end{definition}



\begin{definition}[Event Log]
    
\end{definition}

\begin{definition}[Translucent Event Log]
\end{definition}

\begin{definition}[Lucency]
    
\end{definition}

\begin{definition}[Directly-Follows Graph]
    
\end{definition}

\begin{definition}[Finally-Follows Graph]
    
\end{definition}

\begin{definition}[Non-local dependency]
    
\end{definition}

\begin{definition}[Mining Non-local dependencies]
    
\end{definition}



\begin{comment}
    Limitations of translucent logs: Cannot solve the problem of non-local dependencies, as there are non-free-choice nets that are not lucent. However: Translucent event logs function as a safeguard to differentiate non-local dependencies from the local dependencies in a potentially incomplete event log.
\end{comment}
